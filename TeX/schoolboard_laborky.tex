\documentclass[12pt]{article}
\usepackage[czech]{babel} % nastavuje české popisky např. u obsahu, referencí, tabulek, obázků 
\usepackage[utf8]{inputenc} % použito UTF8 kvůli češtině (zvládá prakticky všechny jazyky na světě)

%\usepackage{indentfirst} % odsazuje první odstavec v kapitole

\usepackage{color} % balíček pro obarvování textů
% \color{blue}
\usepackage{xcolor}  % zapne možnost používání barev, mj. pro \definecolor
\definecolor{mygreen}{RGB}{0,150,0} % nastavení barev odkazů 
\definecolor{myblue}{RGB}{0,0,200} 
\definecolor{commentgreen}{RGB}{0,100,0} % nastavení barev pro příklady z C++
\definecolor{deepblue}{rgb}{0,0,0.7}
\definecolor{deepred}{rgb}{0.6,0,0}
\definecolor{deepgreen}{rgb}{0,0.5,0}

\usepackage{hyperref} % balíček pro hypertextové odkazy
% \url{www.odkaz.cz}
% \href{http://www.odkaz.cz}{Text který bude jako odkaz}
%\hyperlink{label}{proklikávací_text} - odkaz na text 
% \hypertarget{label}{cíl_odkazu} - cíl odkazu  


\hypersetup{colorlinks=true, linkcolor=myblue, urlcolor=mygreen, citecolor=blue, anchorcolor = magenta,
	linktocpage = true, frenchlinks } % nastavení barvy odkazů 
% bookmarksopen=true, bookmarksnumbered=true, bookmarksopenlevel=1 - nastavuje rozbalování levého menu       

\usepackage{graphicx} % pro vkládání obrázků a příkaz "\includegraphics"
%% samotné vložení obrázku
%\begin{figure}
%	\includegraphics[width=\textwidth]{../img/when_use_latex.png}
%	\caption{Kdy se vyplatí použít \LaTeX} % popis, který se zobrazí pod obrázkem
%	\label{moje_navesti} %identifikuje objekt, který lze pak referencovat
%\end{figure}
%% ---------------------------

% Délky a mezery pro celý dokument
\setlength{\parskip}{0.5ex plus 0.5ex minus 0.2ex} %Nastavuje velikost vertikální mezery mezi odstavci:
% první číslo je mezera mezi odstavci, druhé nevím a třetí je mezera před začátkem nové kapitoly a podkapitoly
\parindent=0pt % odstavce nebudou odsazeny zarážkou (velikost odsazení prvního řádku odstavce)
\usepackage{enumitem} % pro příkaz \setlist
\setlist{itemsep = -1pt, topsep=3pt} % nastavuje mezery mezi a před výčtovým prosředím (enumerate, itemize ...)

%\newcommand{\par}{\vspace{0.5\baselineskip}\textbf }
%%%%%%%%%%%%%%%%%%%%%%%%%%%%%%%%%%%%%%%%%%%%%%%%%%%%%%%%%%%%%%%%%%%%%%%%%
\begin{document}

\begin{center}
	\large \textbf{Laboratorní práce} \\  
	\Huge \textbf{Digitální technika} 
\end{center}

\section{Úvod}

Postup práce: 
\begin{itemize}
	\item Představení desky Schoolboard a jejích součástí. 
	\item Představení programu Visual Studio Code, program i s rozšířením PlatformIO je potřeba mít pro studenty nainstalovaný předem. 
	\item Ukázka nahrání programu (\textit{test.h}) do čipu na desce a vysvětlení významu jednotlivých příkazů. 
	\item Vzory pro jednotlivé programy si studenti vždy nahrají do čipu na desce a potom je zkusí upravit podle 
	zadání.\footnote{Přitom je potřeba dodržovat pojmenování proměnných i souborů bez diakritiky a bez mezer 
	a počítat s tím že v jazyku C++ se rozlišují velká a malá písmena. } 
	
\end{itemize}

\section{Blikání LED}

\begin{itemize} 
	\item Do čipu nahrejte program \textit{blink.h}. 
	\item Upravte program tak, aby blikala zelená LED s periodou 0,5 s. 
\end{itemize} 

\section{Čtení tlačítka a výpis do sériové linky}

\begin{itemize} 
	\item Do čipu nahrejte program \textit{button\_serial.h}.
	\item Otevřete terminál a sledujte činnost programu.  
	\item Ukončete terminál můžete stiskem Ctrl+C, napřed je potřeba kliknout myší do prostoru terminálu. 
	Vždy před nahráním programu do čipu je potřeba ukončit terminál!
	\item Upravte program tak, aby posílal data do terminálu přiměřenou rychlostí, např. 2x za sekundu. 
\end{itemize} 

\section{Počítání stisků tlačítka} 

\begin{itemize} 
	\item Do čipu nahrejte program \textit{button\_counting.h}. 
	\item Stiskněte opakovaně tlačítko 0 a sledujte výpis na terminálu. Co pozorujete? 
	\item Upravte program tak, aby se při krátkém stisku tlačítka zvedla hodnota na terminálu vždy o 1.
\end{itemize} 


\end{document}
	*********************************************************************************************

button\_LED

\section{NĚKAM SE MUSÍ NAPSAT ŘEŠENÍ a pokyny pro učitele}


	\section{Rozsvícení LED na stisk tlačítka}




\begin{itemize}
	
	\item rozsviť LED na x sekund při změně stavu tlačítka 
	\item čtení ultrazvuku 
	\item rozsvícení LED podle ultrazvuku 
	\item regulace jasu LED pomocí PWM - přímo čtyřstupňová
	\item regulace jasu LED pomocí PWM - na stisky tlačítka1 se postupně rozsvěcí a na stisk tlačítka2 postupně zhasíná 
	
	
	\item posílání dat na sériovou linku - rozsvítit LED na ALKS
	\item čtení dat ze sériové linky - hodnoty z potenciometru pro 
	
	\item  infra - detekce 
	
	
	\item zprovoznit I2C ? více zařízení na I2C ???? 
	
\end{itemize}

- jaké máme senzory? (6x) infra, ultra, 



- změřit na oscil, co z toho leze 

